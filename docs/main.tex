%%%%%%%%%%%%%%%%%%%%%%%%%%%%%%%%%%%%%%%%%%%%%%%%%%%%%%%%%%%%%%%%%%%%%%%%%%%%%%%%
%2345678901234567890123456789012345678901234567890123456789012345678901234567890
%        1         2         3         4         5         6         7         8

\documentclass[letterpaper, 10 pt, conference]{ieeeconf}  % Comment this line out
                                                          % if you need a4paper
%\documentclass[a4paper, 10pt, conference]{ieeeconf}      % Use this line for a4
                                                          % paper

\IEEEoverridecommandlockouts                              % This command is only
                                                          % needed if you want to
                                                          % use the \thanks command
\overrideIEEEmargins
% See the \addtolength command later in the file to balance the column lengths
% on the last page of the document

\usepackage{xcolor}


% The following packages can be found on http:\\www.ctan.org
%\usepackage{graphics} % for pdf, bitmapped graphics files
%\usepackage{epsfig} % for postscript graphics files
%\usepackage{mathptmx} % assumes new font selection scheme installed
%\usepackage{times} % assumes new font selection scheme installed
%\usepackage{amsmath} % assumes amsmath package installed
%\usepackage{amssymb}  % assumes amsmath package installed

\title{\LARGE \bf
Infertense (Name Subject to Future Punny Change)
}

%\author{ \parbox{3 in}{\centering Huibert Kwakernaak*
%         \thanks{*Use the $\backslash$thanks command to put information here}\\
%         Faculty of Electrical Engineering, Mathematics and Computer Science\\
%         University of Twente\\
%         7500 AE Enschede, The Netherlands\\
%         {\tt\small h.kwakernaak@autsubmit.com}}
%         \hspace*{ 0.5 in}
%         \parbox{3 in}{ \centering Pradeep Misra**
%         \thanks{**The footnote marks may be inserted manually}\\
%        Department of Electrical Engineering \\
%         Wright State University\\
%         Dayton, OH 45435, USA\\
%         {\tt\small pmisra@cs.wright.edu}}
%}

\author{Nice Peter}

% Comments
\newcommand{\nitin}[1]{\color{cyan} Nitin: #1\color{black}}


\begin{document}

\maketitle
\thispagestyle{empty}
\pagestyle{empty}


%%%%%%%%%%%%%%%%%%%%%%%%%%%%%%%%%%%%%%%%%%%%%%%%%%%%%%%%%%%%%%%%%%%%%%%%%%%%%%%%
\begin{abstract}
    We consider a static type system for checking the partial correctness
    of tensor operations using numpy ndarrays in Python. We define the 
    calculus for this type system, and implement it as an extension of the
    Pyre typechecker. We then attempt to quantify the usefulness of our type
    system by running it against existing Python codebases on Github 
    (\nitin{More specifics needed here.}).
    \end{abstract}


%%%%%%%%%%%%%%%%%%%%%%%%%%%%%%%%%%%%%%%%%%%%%%%%%%%%%%%%%%%%%%%%%%%%%%%%%%%%%%%%

\section{INTRODUCTION}
    Python is a dynamically typed general purpose programming language 
    that has gained popularity due to its simplicity and conciseness.
    However, without the benefits of static typing, it becomes easy for
    a programmer to shoot themself in the foot. The release of Python3.5
    introduced type hints into the AST, which galvanized the development 
    of several third party typecheckers for Python (MyPy, Pyre).

\section{SYNTAX AND SEMANTICS}
\subsection{Relevant Tensor Operations}
\subsection{Tensor Calculus}
\subsection{Abstract Interpretation in Pyre}
\subsection{Type System}
\subsection{Theorems?}

\section{Evaluation}

\section{Conclusion}


\bibliographystyle{unsrt}
\bibliography{sample}

\end{document}
